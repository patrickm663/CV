%%%%%%%%%%%%%%%%%
% This is an sample CV template created using altacv.cls
% (v1.3, 10 May 2020) written by LianTze Lim (liantze@gmail.com). Now compiles with pdfLaTeX, XeLaTeX and LuaLaTeX.
% (v1.6.5b, 27 Jun 2023) forked by Nicolás Omar González Passerino (nicolas.passerino@gmail.com)
%
%% It may be distributed and/or modified under the
%% conditions of the LaTeX Project Public License, either version 1.3
%% of this license or (at your option) any later version.
%% The latest version of this license is in
%%    http://www.latex-project.org/lppl.txt
%% and version 1.3 or later is part of all distributions of LaTeX
%% version 2003/12/01 or later.
%%%%%%%%%%%%%%%%

%% If you need to pass whatever options to xcolor
\PassOptionsToPackage{dvipsnames}{xcolor}

%% If you are using \orcid or academicons
%% icons, make sure you have the academicons
%% option here, and compile with XeLaTeX
%% or LuaLaTeX.
% \documentclass[10pt,a4paper,academicons]{altacv}

%% Use the "normalphoto" option if you want a normal photo instead of cropped to a circle
% \documentclass[10pt,a4paper,normalphoto]{altacv}

%% Fork (before v1.6.5a): CV dark mode toggle enabler to use a inverted color palette.
%% Use the "darkmode" option if you want a color palette used to 
% \documentclass[10pt,a4paper,ragged2e,withhyper,darkmode]{altacv}

\documentclass[10pt,a4paper,ragged2e,withhyper]{altacv}

%% AltaCV uses the fontawesome5 and academicons fonts
%% and packages.
%% See http://texdoc.net/pkg/fontawesome5 and http://texdoc.net/pkg/academicons for full list of symbols. You MUST compile with XeLaTeX or LuaLaTeX if you want to use academicons.

% Change the page layout if you need to
\geometry{left=1cm,right=1cm,top=1cm,bottom=1cm,columnsep=0.75cm}

% The paracol package lets you typeset columns of text in parallel
\usepackage{paracol}

% Change the font if you want to, depending on whether
% you're using pdflatex or xelatex/lualatex
\ifxetexorluatex
  % If using xelatex or lualatex:
  \setmainfont{Roboto Slab}
  \setsansfont{Lato}
  \renewcommand{\familydefault}{\sfdefault}
\else
  % If using pdflatex:
  \usepackage[rm]{roboto}
  \usepackage[defaultsans]{lato}
  % \usepackage{sourcesanspro}
  \renewcommand{\familydefault}{\sfdefault}
\fi

% Fork (before v1.6.5a): Change the color codes to test your personal variant on any mode
\ifdarkmode%
  \definecolor{PrimaryColor}{HTML}{C69749}
  \definecolor{SecondaryColor}{HTML}{D49B54}
  \definecolor{ThirdColor}{HTML}{1877E8}
  \definecolor{BodyColor}{HTML}{ABABAB}
  \definecolor{EmphasisColor}{HTML}{ABABAB}
  \definecolor{BackgroundColor}{HTML}{FFFFFF}
\else%
  \definecolor{PrimaryColor}{HTML}{001F5A}
  \definecolor{SecondaryColor}{HTML}{0039AC}
  \definecolor{ThirdColor}{HTML}{F3890B}
  \definecolor{BodyColor}{HTML}{666666}
  \definecolor{EmphasisColor}{HTML}{2E2E2E}
  \definecolor{BackgroundColor}{HTML}{FFFFFF}
\fi%

\colorlet{name}{PrimaryColor}
\colorlet{tagline}{SecondaryColor}
\colorlet{heading}{PrimaryColor}
\colorlet{headingrule}{ThirdColor}
\colorlet{subheading}{SecondaryColor}
\colorlet{accent}{SecondaryColor}
\colorlet{emphasis}{EmphasisColor}
\colorlet{body}{BodyColor}
\pagecolor{BackgroundColor}

% Change some fonts, if necessary
\renewcommand{\namefont}{\Huge\rmfamily\bfseries}
\renewcommand{\personalinfofont}{\small\bfseries}
\renewcommand{\cvsectionfont}{\LARGE\rmfamily\bfseries}
\renewcommand{\cvsubsectionfont}{\large\bfseries}

% Change the bullets for itemize and rating marker
% for \cvskill if you want to
\renewcommand{\itemmarker}{{\small\textbullet}}
\renewcommand{\ratingmarker}{\faCircle}


\begin{document}
    \name{Patrick Moehrke}
    \tagline{Technical Consultant | Cloud Practitioner}

    \personalinfo{
        \email{parickmoehrke46@gmail.com}\smallskip
        \location{Somerset West, Western Cape, ZA}\\
        \linkedin{patrick-m-0509ab3}\smallskip
        \github{patrickm663}\smallskip
        \homepage{patrickmoehrke.com}
        %\medium{nicolasomar}
        %% You MUST add the academicons option to \documentclass, then compile with LuaLaTeX or XeLaTeX, if you want to use \orcid or other academicons commands.
        % \orcid{0000-0000-0000-0000}
        %% You can add your own arbtrary detail with
        %% \printinfo{symbol}{detail}[optional hyperlink prefix]
        % \printinfo{\faPaw}{Hey ho!}[https://example.com/]
        %% Or you can declare your own field with
        %% \NewInfoFiled{fieldname}{symbol}[optional hyperlink prefix] and use it:
        % \NewInfoField{gitlab}{\faGitlab}[https://gitlab.com/]
        % \gitlab{your_id}
    }
    
    \makecvheader
    %% Depending on your tastes, you may want to make fonts of itemize environments slightly smaller
    \AtBeginEnvironment{itemize}{\small}
    
    %% Set the left/right column width ratio to 6:4.
    \columnratio{0.28}

    % Start a 2-column paracol. Both the left and right columns will automatically
    % break across pages if things get too long.
    \begin{paracol}{2}
        % ----- SKILLS -----
        \cvsection{Skills}
	\cvsubsection{Tech Stack}
            \cvtag{Python}
            \cvtag{R}
            \cvtag{Julia}
            \cvtag{SQL}
            \cvtag{Lua}
            \cvtag{D}
            \cvtag{Bash}
            \cvtag{JavaScript}\\
            %\medskip
            \cvtag{Git}
            \cvtag{Docker}
            \cvtag{Linux}
            \cvtag{\LaTeX}
            \cvtag{GitHub Actions}
            \cvtag{GitLab CICD}
            \cvtag{EC2}
            \cvtag{Lambda}
            \cvtag{S3}
            \cvtag{AWS Lambda}
            \cvtag{Terraform}
        
	\cvsubsection{Techniques}
            \cvtag{Explainable ML}\\
            \cvtag{Probabilistic Programming}
            \cvtag{Containerisation}
            \cvtag{MLOps}
            \cvtag{Serverless Computing}\\
            \cvtag{High-performance Computing}
            \cvtag{Version Control}
            \cvtag{Solution Architecture}\\
            \cvtag{Technical Documentation}\\

	\cvsubsection{Professional Skills}
            \cvtag{Technical Writing}
            \cvtag{Mentoring}
            \cvtag{Stakeholder Engagement}
            \cvtag{Innovative}
            \cvtag{Critical Thinking}
            \cvtag{Collaboration}
        % ----- SKILLS -----

	% ----- AWARDS & CERTIFICATIONS -----
	\cvsection{Awards \& \\ Certifications}
	  \cvachievement{AWS Certified Cloud Practitioner CLF-C01}{April 2023}
	  \cvachievement{Technical Actuarial Member of ASSA (TASSA)}{July 2022}
	  \cvachievement{Member of Golden Key International Honours Society}{September 2019}

	% ----- AWARDS & CERTIFICATIONS -----
            
        % ----- REFERENCES -----
        \cvsection{References}
            \cvref{Person A}{Institute}{a.beta@university.edu}
            \cvref{Person B}{Business}{g.delta@business.com}
        % ----- REFERENCES -----
        
        
        \newpage
        
        %% Switch to the right column. This will now automatically move to the second
        %% page if the content is too long.
        \switchcolumn
        
        % ----- PROFILE -----
        \cvsection{Profile}
            \begin{quote}
                I am an aspiring DevOps engineer with demonstrated experience in XXX.\\

		I am a self-started who values open source and continuous learning, with my current focus being cloud native techniques.
            \end{quote}
        % ----- PROFILE -----
        
        % ----- EXPERIENCE -----
        \cvsection{Experience}
            \cvevent{Actuarial Consultant}{Dupro Advisory}{July 2020 -- Present}{Stellenbosch, ZA}
            \begin{itemize}
	      \item Deliver training material \& workshops on machine learning and cloud to insurance and financial service providers
	      \item On a day-to-day basis, I use a range of languages and environemnts as appropriate. 
              \item Third achievement
            \end{itemize}
        
            \divider
            \cvevent{Academic Tutor}{Stellenbosch University}{January 2020 -- July 2020}{Stellenbosch, ZA}
            \begin{itemize}
	      \item I served as an academic tutor for Actuarial Science 112 students. this included mentoring, explaining complex topics, and grading assignments. 
            \end{itemize}
            
        % ----- EXPERIENCE -----
        
        % ----- EDUCATION -----
        \cvsection{Education}
            \cvevent{BCom Honours Actuarial Science}{Stellenbosch University}{Jan 2020 -- Dec 2020}{Stellenbosch, ZA}
	    Completed coursework and a research project ('Evaluating the use of genetics to calculate risk premiums with applications to Huntington's Disease') whilst leading a campus society. The core competencies included actuarial risk management, financial engineering, and statistical programming.

            \divider
            
            \cvevent{BCom Actuarial Science}{Stellenbosch University}{Jan 2017 -- Dec 2019}{Stellenbosch, ZA}
	    Well-rounded student with demonstrated academic performance and involvement in leading campus societies. Subject areas included actuarial science, statistics, computer science, mathematics, and economics.

            \divider

            \cvevent{IEB National Senior Certificate}{Beaulieu College}{Jan 2012 -- Dec 2016}{Johannesburg, ZA}
	      IEB matric certificate with an A average. Awarded top achiever in 3 subjects.
        % ----- EDUCATION -----
        
        % ----- PROJECTS -----
        \cvsection{Projects}
            \cvevent{Project 1 }{\cvreference{\faGithub}{https://github.com/user/repo}\cvreference{| \faGlobe}{https://project-demo.com/}}{Mm YYYY -- Mm YYYY}{}
            \begin{itemize}
                \item Item 1
                \item Item 2
            \end{itemize}
            \divider
            
            \cvevent{Project 2 }{\cvreference{\faGitlab}{https://gitlab.com/user/repo}\cvreference{| \faGlobe}{https://project-demo.com/}}{Mm YYYY -- Mm YYYY}{}
            \begin{itemize}
                \item Item 1
                \item Item 2
            \end{itemize}
        % ----- PROJECTS -----
    \end{paracol}
\end{document}
